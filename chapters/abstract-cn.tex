% 随着毫米波雷达点云成像技术在自动驾驶、医疗成像、地形测绘等领域的大量使用,毫米波雷达点云检测与多点云配准作为重要步骤,其精确度有了更高的要求。然而,在低信噪比、多目标的情况下,使用恒虚警率(CFAR)算法检测毫米波雷达点云时,存在性能退化和无法有效检测出目标点的问题。此外,由于毫米波雷达点云具有低分辨率、稀疏性、密度分布不均匀的特点,存在无法有效提取点特征和点云较稠密处提取的特征在稀疏处不适用的问题,传统的点云配准方法在其应用中面临严峻挑战。因此,本文聚焦毫米波雷达点云检测与配准,具体研究工作如下:

% 毫米波雷达点云成像技术在自动驾驶、医疗成像、地形测绘等领域发挥着重要作用,雷达目标点检测与多点云配准是其重要组成部分。由于毫米波雷达工作频率低、噪声干扰大,其点云具有低分辨率、稀疏性、密度分布不均匀的特点,这些特点导致无法有效检测毫米波雷达目标点且多毫米波雷达点云难以精确配准。本文聚焦毫米波雷达点云检测与配准,基于毫米波雷达距离多普勒图特征和多毫米波雷达点云多尺度邻域特征,研究毫米波雷达目标点检测方案和毫米波雷达点云精确配准方案,有效提升了目标点检测准确度与多毫米波雷达点云配准精度。

毫米波雷达点云成像技术在自动驾驶、医疗成像、地形测绘等领域发挥着重要作用,雷达目标点检测与多点云配准是其重要组成部分。然而,由于毫米波雷达的工作频率较低且易受噪声干扰,其产生的点云分辨率较低、稀疏且密度分布不均,导致难以有效检测目标点和多点云难以精确配准。本文聚焦毫米波雷达点云检测与配准,基于毫米波雷达距离多普勒图特征和多毫米波雷达点云多尺度邻域特征,研究毫米波雷达目标点检测方案和毫米波雷达点云精确配准方案,有效提升了目标点检测准确度与多毫米波雷达点云配准精度。

\par
针对恒虚警率检测算法在低信噪比、多目标的情况下无法有效检测出目标点的问题,提出一种基于残差神经网络的距离多普勒图目标点检测方案RA-CFAR。通过多层残差块提取距离多普勒图全局特征进行全局噪声评估,引入了自注意力模块捕捉距离多普勒图峰值区域的空间位置相关性,克服距离多普勒图中目标点位置的旁瓣效应。在公开数据集以及自建数据集上将本方案与多种常用恒虚警率检测算法进行对比,结果表明,相较于CA-CFAR、SOCA-CFAR、GOCA-CFAR、OS-CFAR,在低信噪比情况下AUC值分别提升了4\%、8\%、11.9\%、29.4\%;在多目标情况下AUC值分别提升了2.29\%、2.71\%、3.36\%、7.19\%。
\par
针对无法有效提取点特征和点云较稠密处提取的特征在稀疏处不适用的问题,提出基于多尺度邻域的特征提取方案。在预处理阶段引入KD树计算邻域内的邻近点辅助拟合点云邻域平面,使用拟合平面法向量作为点云法向量的方式增加邻域信息丰富度。在特征提取阶段为了提升特征在不同稀疏区域的适用性,结合多普勒信息、拟合法向量和空间坐标信息,在多个$\tau$尺寸的邻域上提取特征合并作为点特征对抗点云的稀疏和不均匀性。在配准阶段使用特征距离代替空间距离增强配准算法的鲁棒性,通过全局特征推测配准初始值,并且在每次迭代后优化配准参数,避免初始值设置不当导致配准难以收敛的问题。实验结果表明,与同类方案相比本方案的 RMSE(R)、MAE(R)、RMSE(T)、MAE(T)分别降低了2.2\%、8.9\%、 20.9\%、35\%。

以上研究内容均在毫米波雷达公开数据集以及自建数据集上通过检测率和配准损失率等评估指标进行了实验验证,实验结果表明,本文提出的毫米波雷达点云配准方案在低信噪比、多目标、稀疏性、密度分布不均匀等情况下均表现出优异的性能。