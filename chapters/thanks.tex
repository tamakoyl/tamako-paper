时光飞逝,转眼间已到毕业之际。在这三年间遇到了许多良师益友,他们对我的工作和学习提供许多帮助,在此,我要向这些人表达最真挚的感谢和敬意。

首先,我要感谢沈玉龙老师,沈老师严谨的科研态度,塑造了良好的实验室氛围;沈老师为实验室提供了优越的实验条件和充足的实验设备,并经常组织学术讲座,极大的开拓了我的专业视野。感谢祝幸辉老师,研究生三年,祝老师亦师亦友,在工作和学习上为我提供了许多关心和帮助;在学习和科研上,祝老师为我指明了方向让我能够继续探索;在生活上他所传授丰富的阅历让我不断地成长。

感谢实验室的所有老师和同学,感谢他们的帮助为我的研究带来了许多启发与灵感。感谢师兄师姐刘太鹏、匡昶、孙俊飞、邱和龙、王凯、王海洋、张翰霆、王娟、严静、张潇丹,为我了科研和工作提供了许多指导,让我少走了许多弯路。感谢同门毛佳、吕龙龙、曹强、冯林林、张启元、田泽中、窦俊、顾书浩、童伟,有他们陪伴、帮助和支持,让我的研究生生活充满乐趣。感谢师弟师妹罗文勋、杨丽、陈鸿林、文金铭、梁家杰、滕少西,感谢他们在我找工作以及写论文时承担项目,让我没有后顾之忧。感谢我的朋友谢亦高、张智嵩、陈健航,在我读研期间的鼓励以及帮助,让我更有勇气坚持下去。

感谢我的家人,在我多年的求学中生涯中,家人的无条件支持和付出永远是我最坚实的后盾和最温暖的港湾。

感谢所有曾为我提供过帮助与支持的人。感谢以上所有提及的人,还要感谢那些虽然没被提及但是对我有过帮助的人,虽未提及但是我会永远记得你们的帮助,谢谢你们!

