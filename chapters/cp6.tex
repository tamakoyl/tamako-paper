\chapter{总结与展望}
\section{本文工作总结}
为了解决毫米波雷达点云在应用过程中的目标点检测与配准问题,本文深入分析了毫米波雷达数据在目标点检测和配准两个阶段的特点和难点,完成了以下工作:
\par
(1) 针对毫米波雷达工作中易受到噪声干扰,而恒虚警率算法在低信噪比、多目标的情况下无法有效检测目标点的问题,提出了一种基于残差神经网络的目标点提取方案。使用多层残差块提取距离多普勒图全局特征,引入十字交叉注意力机制着重关注距离多普勒图中目标点所在位置行和列的特征,帮助模型更有效地提取位置信息捕捉空间位置关联性。与其他常用算法通过实验对比,验证了方案的适用性和有效性。
\par
(2) 针对毫米波雷达点云密度稀疏且不均匀,导致传统配准算法难以工作的问题,提出使用多尺度邻域的特征提距离代替空间距离的方案,并使用拟合法向量、多普勒信息提取等方式进一步丰富了邻域特征。为了避免初始值设置不当导致配准难以收敛的问题,通过全局特征推测配准初始值并且在每次迭代后优化配准参数。实验结果表明,与同类方案相比本方案的 RMSE(R)、MAE(R)、RMSE(T)、MAE(T)分别降低了2.2\%、8.9\%、 20.9\%、35\%。
\par
(3) 为了验证本文提出方案在落地场景中的有效性,搭建了原型系统并且集成进现有的校园安防平台,用于多毫米波雷达场景的毫米波雷达点云数据融合,并且用融合后的数据进行目标检测和跟踪。验证了本方案的有效性和实用性。
\section{未来工作展望}
本文针对毫米波雷达距离多普勒图目标点检测和点云配准问题进行了探索,给出了一些解决方案,但是还有很多工作可以继续深入研究:
\par
(1) 本文采取了分析距离多普勒图的方式进行目标点提取,考虑到多帧距离多普勒图在时间序列上具有连续性,因此未来可以考虑使用时序模型结合当前帧的距离多普勒图进行目标点提取。
\par
(2) 本文提出了基于多尺度邻域的特征提取方案,但是在实际应用中,如果选择的邻域半径过多,会导致特征提取方案的计算量较大,因此未来需要一个邻域半径选择方案,辅助多尺度邻域特征提取。