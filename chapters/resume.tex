\section{基本情况} \anon{赵文超},\anon{男},\anon{河南周口人},\anon[XXX ]{1997年12月出生},\anon[XXX ]{西安电子科技大学}\anon[XXX ]{广州研究院}\anon{计算机技术}专业2021级硕士研究生。 
\section{教育背景}
\begin{edubg}
\anon{2017.09~2021.07} & \anon{河南工业大学},本科,专业:\anon{软件工程}\\
\anon{2021.09~ }& \anon{西安电子科技大学},硕士研究生,专业:\anon{计算机技术}\\
\end{edubg}
\section{攻读硕士学位期间的研究成果}
\subsection{申请(授权)专利}
\begin{resresult}
	\item \anon[(第一学生发明人)]{祝幸辉, \textbf{\anon{赵文超}}, 沈玉龙等}.一种基于深度学习的毫米波雷达点云数据迭代配准方法:中国, CN202410300316.6
\end{resresult}

\subsection{参与科研项目及获奖}
\begin{resresult}
	\item 国家重点研发计划, 智慧城市云计算平台及服务关键技术研究, 2019.12-2022.11, 结题, 参与。
	\item 国家自然科学基金, 多域物联网系统数据安全关键技术研究, 2023.01-2027.12, 在研, 参与。
	\item 企、事业单位委托项目, 校园安防智慧哨兵系统, 2021.06-2024.07, 结题, 研发负责人。

\end{resresult}
