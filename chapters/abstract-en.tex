Millimeter-wave radar point cloud imaging technology plays a significant role in fields such as autonomous driving, medical imaging, and terrain mapping. Radar target point detection and multi-point cloud registration are crucial steps in its application. Due to the low working frequency and high noise interference of millimeter-wave radar, its point clouds are characterized by low resolution, sparsity, and uneven density distribution. These characteristics lead to ineffective detection of millimeter-wave radar target points and difficulties in precise registration of multiple millimeter-wave radar point clouds. This thesis focuses on millimeter-wave radar point cloud detection and registration. Based on the features of millimeter-wave radar range-Doppler maps and the multi-scale neighborhood features of multiple millimeter-wave radar point clouds, this study explores millimeter-wave radar target point detection schemes and precise registration schemes for millimeter-wave radar point clouds, effectively improving the accuracy of target point detection and the precision of multi-millimeter-wave radar point cloud registration.

Addressing the issue that the Constant False Alarm Rate (CFAR) algorithm cannot effectively detect target points under low signal-to-noise ratios and multiple target scenarios, this thesis proposes a Distance-Doppler map target point detection scheme based on Residual Neural Networks, named RA-CFAR. Through extracting global features from distance-Doppler maps using multi-layer residual blocks for global noise assessment, this scheme introduces a self-attention module to capture the spatial location relevance of the peak areas in the distance-Doppler maps, overcoming the sidelobe effects of target point locations within the maps. Comparisons with CA-CFAR、SOCA-CFAR、GOCA-CFAR and OS-CFAR on public datasets and proprietary datasets demonstrate that, in low signal-to-noise ratio scenarios, the Area Under the Curve (AUC) values increased by 4\%, 8\%, 11.9\%, and 29.4\% respectively; in multi-target scenarios, the AUC values increased by 2.2\%, 2.71\%, 3.36\%, and 7.19\% respectively.

In response to the problem of ineffective point feature extraction and the inapplicability of features extracted in denser areas of point clouds in sparser areas, this thesis proposes a feature extraction scheme based on multi-scale neighborhoods. In the preprocessing stage, a KD-tree is introduced to calculate nearby points within the neighborhood to assist in fitting the plane of the point cloud neighborhood. The fitted plane's normal vector is used as the point cloud's normal vector to enrich neighborhood information. During the feature extraction stage, to improve the applicability of features in different sparse regions, features are extracted and combined as point characteristics on multiple neighborhoods of different $\tau$ sizes, integrating Doppler information, fitted normal vectors, and spatial coordinates. In the registration phase, feature distance replaces spatial distance to enhance the robustness of the registration algorithm. Global features are used to estimate initial registration values, and registration parameters are optimized after each iteration to avoid issues with registration convergence caused by improper initial value settings. Experimental results demonstrate that, compared to similar schemes, the proposed solution has reduced RMSE(R), MAE(R), RMSE(T), and MAE(T) by 2.2\%, 8.9\%, 20.9\%, and 35\%, respectively.

The research content discussed has been experimentally validated on public millimeter-wave radar datasets and proprietary datasets through evaluation metrics such as detection rate and registration loss rate. The results indicate that the millimeter-wave radar point cloud registration scheme proposed in this thesis exhibits excellent performance in scenarios with low signal-to-noise ratios, multiple targets, sparsity, and uneven density distribution.